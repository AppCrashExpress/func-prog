\documentclass[12pt]{article}

\usepackage{fullpage}
\usepackage{multicol,multirow}
\usepackage{tabularx}
\usepackage{ulem}
\usepackage[utf8]{inputenc}
\usepackage[russian]{babel}
\usepackage{amsmath}
\usepackage{amssymb}

% For source code output
\usepackage{xcolor}
\usepackage{minted}

\usepackage{titlesec}

\titleformat{\section}
  {\normalfont\Large\bfseries}{\thesection.}{0.3em}{}

\titleformat{\subsection}
  {\normalfont\large\bfseries}{\thesubsection.}{0.3em}{}

\titlespacing{\section}{0pt}{*2}{*2}
\titlespacing{\subsection}{0pt}{*1}{*1}
\titlespacing{\subsubsection}{0pt}{*0}{*0}
\usepackage{listings}
\lstloadlanguages{Lisp}
\lstset{extendedchars=false,
	breaklines=true,
	breakatwhitespace=true,
	keepspaces = true,
	tabsize=2
}
\begin{document}


\section*{Отчет по лабораторной работе №\,1 
по курсу \guillemotleft  Функциональное программирование\guillemotright}
\begin{flushright}
Студент группы 8О-307 МАИ \textit{Ефимов Александр}, \textnumero 7 по списку \\
\makebox[7cm]{Контакты: {\tt aleks.efimov2011@yandex.ru} \hfill} \\
\makebox[7cm]{Работа выполнена: 27.02.2021 \hfill} \\
\ \\
Преподаватель: Иванов Дмитрий Анатольевич, доц. каф. 806 \\
\makebox[7cm]{Отчет сдан: \hfill} \\
\makebox[7cm]{Итоговая оценка: \hfill} \\
\makebox[7cm]{Подпись преподавателя: \hfill} \\

\end{flushright}

\section{Тема работы}
Примитивные функции и особые операторы Common Lisp.

\section{Цель работы}
Научиться вводить S-выражения в Lisp-систему, определять переменные и функции, работать с условными
операторами, работать с числами, используя схему линейной и древовидной рекурсии.

\section{Задание (вариант №1.45)}
С помощью формулы

$$
\frac{1}{3} \left(\frac{x}{y^2} + 2y\right) 
$$

запрограммируйте на языке Common Lisp функцию для вычисления кубического корня. Причем y является
приближением к кубическому корню из х.

Использовать функции {\tt good-enough-p}, {\tt improve} и {\tt cube}.

\section{Оборудование студента}
Процессор Intel(R) Core(TM) i5-8250U CPU @ 1.60GHz, память: 7.6Gi, разрядность системы: 64.

\section{Программное обеспечение}
ОС Arch Linux, утилита CLisp.

\section{Идея, метод, алгоритм}
Формула, предоставленная в задании, выводится из метода Ньютона, а именно: необходимо найти
некоторый $x$ \--- кубический корень числа $a$ такой, что $x^3 = a$. Если это обозначить за 
функцию: 
$$
f(x) = x^3 - a
$$
то с помощью метода Ньютона можно итеративно найти кубический корень:

$$
x_{k + 1} = x_{k} - \frac{f(x_{k})}{f'(x_{k})} 
          = x_{k} - \frac{x_{k}^3 - a}{3x_{k}^2}
          = \frac{3x_{k}^3 - x_{k}^3 + a}{3x_{k}^2} 
          = \frac{1}{3}\left(2x_{k} + \frac{a}{x_{k}^2}\right)
$$

Каждое новое значение $x_{k + 1}$ может иметь тип \textit{float}, поэтому его нужно сравнивать
с оригиналом в пределах машинного (или заданного разработчиком) эпсилона.

% \section{Сценарий выполнения работы}

\section{Распечатка программы и её результаты}

\subsection{Исходный код}
% \lstinputlisting{./cuberoot.lisp}
\inputminted[linenos, frame=lines]{lisp}{./cuberoot.lisp}

\subsection{Результаты работы}
% \lstinputlisting{./log.lisp}
\inputminted[frame=lines]{lisp}{./log.lisp}

% \section{Дневник отладки}
% \begin{tabular}{|c|c|c|c|}
% \hline
% Дата & Событие & Действие по исправлению & Примечание \\
% \hline
% \end{tabular}

\section{Замечания автора по существу работы}
Сложность работы искусственно увеличивается ввиду её плохой описанности. Следует
хотя бы упомянуть используемый для решении метод Ньютона для упрощения.

\section{Выводы}
При итеративном делении нужно быть аккуратным не передавать переменные с типом \textit{rational},
иначе можно быстро получить ошибку переполнения.

\end{document}

