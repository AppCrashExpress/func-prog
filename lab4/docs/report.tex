\documentclass[12pt]{article}

\usepackage{fullpage}
\usepackage{multicol,multirow}
\usepackage{tabularx}
\usepackage{ulem}
\usepackage[utf8]{inputenc}
\usepackage[russian]{babel}
\usepackage{amsmath}
\usepackage{amssymb}

% For source code output
\usepackage{xcolor}
\usepackage{minted}

\usepackage{titlesec}

\titleformat{\section}
  {\normalfont\Large\bfseries}{\thesection.}{0.3em}{}

\titleformat{\subsection}
  {\normalfont\large\bfseries}{\thesubsection.}{0.3em}{}

\titlespacing{\section}{0pt}{*2}{*2}
\titlespacing{\subsection}{0pt}{*1}{*1}
\titlespacing{\subsubsection}{0pt}{*0}{*0}
\usepackage{listings}
\lstloadlanguages{Lisp}
\lstset{extendedchars=false,
	breaklines=true,
	breakatwhitespace=true,
	keepspaces = true,
	tabsize=2
}
\begin{document}


\section*{Отчет по лабораторной работе №\,4 
по курсу \guillemotleft  Функциональное программирование\guillemotright}
\begin{flushright}
Студент группы 8О-307 МАИ \textit{Ефимов Александр}, \textnumero 7 по списку \\
\makebox[7cm]{Контакты: {\tt aleks.efimov2011@yandex.ru} \hfill} \\
\makebox[7cm]{Работа выполнена: 02.04.2021 \hfill} \\
\ \\
Преподаватель: Иванов Дмитрий Анатольевич, доц. каф. 806 \\
\makebox[7cm]{Отчет сдан: \hfill} \\
\makebox[7cm]{Итоговая оценка: \hfill} \\
\makebox[7cm]{Подпись преподавателя: \hfill} \\

\end{flushright}

\section{Тема работы}
Знаки и строки

\section{Цель работы}
Научиться работать с литерами (знаками) и строками при помощи функций 
обработки строк и общих функций работы с последовательностями.

\section{Задание}

Вариант: №4.34

Запрограммировать на языке Коммон Лисп функцию, принимающую один аргумент - 
текст.

Если в тексте нет знака $*$, то функция должна вернуть этот текст без изменения. 
В противном случае функция должна вернуть копию текста, в котором все малые 
буквы, предшествующие первому вхождению $*$, заменены на цифру 3.

Функция должна работать как для малых латинских, так и малых русских букв.

\section{Оборудование студента}
Процессор Intel(R) Core(TM) i5-8250U CPU @ 1.60GHz, память: 7.6Gi, разрядность 
системы: 64.

\section{Программное обеспечение}
ОС Arch Linux, система CLisp.

\section{Идея, метод, алгоритм}

Сначала реализовать функцию, которая ищет в строки позицию первого вхождения $*$. 
Если она не найдена, вернуть строку. Иначе, разбить строку на две подстроки 
до и с $*$ в начале. После замены в первой строки всех букв нижнего регистра
на 3 вернуть объединение этой подстроки со второй (где $*$).

Обобщить эту функцию для текста: найти предложение, где содержится $*$ и 
сделать объединение трех списков
\begin{enumerate}
    \item Список предложений до $*$, где все буквы нижнего регистра 
    заменены на 3;
    \item Список с предложением с $*$, обработанным ранее описанной функцией;
    \item Список со всеми оставшимися предложениями.
\end{enumerate}

Предложение с $*$ необходимо обернуть в список для корректной работы функции 
\textit{concatenate}.

\section{Сценарий выполнения работы}

\section{Распечатка программы и её результаты}

\subsection{Исходный код}
\inputminted[linenos, frame=lines]{lisp}{./editor.lisp}

\subsection{Результаты работы}
\inputminted[frame=lines]{lisp}{./log.lisp}

\section{Дневник отладки}
\begin{tabular}{|c|c|c|c|}
\hline
Дата & Событие & Действие по исправлению & Примечание \\
\hline
\end{tabular}

\section{Замечания автора по существу работы}

\section{Выводы}
% \begin{minted}{text}
% Take me down to the Haskell city
% Where the brackets are little
% And syntax is pretty
% Oh won't you please take me home?
% \end{minted}

\begin{itemize}
    \item Манипуляции над строками облегчаются за счет применения
    функций \textit{map} и \textit{position};
    \item Строки можно разбивать на подстроки и работать с ними
    без ограничения общности.
\end{itemize}

\end{document}

