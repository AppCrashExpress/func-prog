\documentclass[12pt]{article}

\usepackage{fullpage}
\usepackage{multicol,multirow}
\usepackage{tabularx}
\usepackage{ulem}
\usepackage[utf8]{inputenc}
\usepackage[russian]{babel}
\usepackage{amsmath}
\usepackage{amssymb}

% For source code output
\usepackage{xcolor}
\usepackage{minted}

\usepackage{titlesec}

\titleformat{\section}
  {\normalfont\Large\bfseries}{\thesection.}{0.3em}{}

\titleformat{\subsection}
  {\normalfont\large\bfseries}{\thesubsection.}{0.3em}{}

\titlespacing{\section}{0pt}{*2}{*2}
\titlespacing{\subsection}{0pt}{*1}{*1}
\titlespacing{\subsubsection}{0pt}{*0}{*0}
\usepackage{listings}
\lstloadlanguages{Lisp}
\lstset{extendedchars=false,
	breaklines=true,
	breakatwhitespace=true,
	keepspaces = true,
	tabsize=2
}
\begin{document}


\section*{Отчет по лабораторной работе №\,5 
по курсу \guillemotleft  Функциональное программирование\guillemotright}
\begin{flushright}
Студент группы 8О-307 МАИ \textit{Ефимов Александр}, \textnumero 7 по списку \\
\makebox[7cm]{Контакты: {\tt aleks.efimov2011@yandex.ru} \hfill} \\
\makebox[7cm]{Работа выполнена: 09.04.2021 \hfill} \\
\ \\
Преподаватель: Иванов Дмитрий Анатольевич, доц. каф. 806 \\
\makebox[7cm]{Отчет сдан: \hfill} \\
\makebox[7cm]{Итоговая оценка: \hfill} \\
\makebox[7cm]{Подпись преподавателя: \hfill} \\

\end{flushright}

\section{Тема работы}
Обобщённые функции, методы и классы объектов

\section{Цель работы}
Научиться определять простейшие классы, порождать экземпляры классов, 
считывать и изменять значения слотов, научиться определять 
обобщённые функции и методы.

\section{Задание}

Вариант: №5.37

Дан экземпляр класса \textit{triangle}, причем все вершины треугольника могут 
быть заданы как декартовыми координатами (экземплярами класса \textit{cart}), 
так и полярными (экземплярами класса \textit{polar}).

Задание: Определить обычную функцию высота, возвращающую объект-отрезок 
(экземпляр класса \textit{line}), являющийся высотой первого угла 
\textit{vertex1}. Концы результирующего отрезка могут быть получены либо в 
декартовых, либо в полярных координатах.

\begin{minted}[linenos, frame=lines]{lisp}
(setq tri (make-instance 'triangle
           :1 (make-instance 'cart-или-polar ...)
           :2 (make-instance 'cart-или-polar ...)
           :3 (make-instance 'cart-или-polar ...)))

(высота tri) =>  [ОТРЕЗОК ...]
\end{minted}

\section{Оборудование студента}
Процессор Intel(R) Core(TM) i5-8250U CPU @ 1.60GHz, память: 7.6Gi, разрядность 
системы: 64.

\section{Программное обеспечение}
ОС Arch Linux, система CLisp.

\section{Идея, метод, алгоритм}

    При вызове функции все точки треугольника считаются однотипными. Если точки
полярные, то перевести их в декартовы координаты.

Пусть нам даны точки треугольника $A, B, C$, причем необходимо найти конец вектора
высоты из точки $A$. Для этого проектируем вектор $\vec{BA}$ на вектор $\vec{BC}$
с помощью скалярного произведения. Полученное произведение даст длину проектированного
сторону $\vec{BC}$. Поделив её на длину самой стороны, можно получить коэффициент,
который после умножения на сторону дает смещение точки $B$ для достижения конца
вектора отрезка высоты $P$. Формульно это будет иметь вид:

$$
    \vec{BA} = A - B
$$
$$
    \vec{BC} = C - B
$$
$$
    coef = (BA \cdot BC) / (BC \cdot BC)
$$
$$
    P = B + coef \cdot BA
$$

\section{Сценарий выполнения работы}

\section{Распечатка программы и её результаты}

\subsection{Исходный код}
\inputminted[linenos, frame=lines]{lisp}{./classes.lisp}

\subsection{Результаты работы}
\inputminted[frame=lines]{lisp}{./log.lisp}

\section{Дневник отладки}
\begin{tabular}{|c|c|c|c|}
\hline
Дата & Событие & Действие по исправлению & Примечание \\
\hline
\end{tabular}

\section{Замечания автора по существу работы}

\section{Выводы}
% \begin{minted}{text}
% Take me down to the Haskell city
% Where the brackets are little
% And syntax is pretty
% Oh won't you please take me home?
% \end{minted}

Классы систематизируют данные, инкапсулируя их. Обобщенные функции позволяют присваиваить одни и те же 
названия схожим действиям.

\end{document}

