\documentclass[12pt]{article}

\usepackage{fullpage}
\usepackage{multicol,multirow}
\usepackage{tabularx}
\usepackage{ulem}
\usepackage[utf8]{inputenc}
\usepackage[russian]{babel}
\usepackage{amsmath}
\usepackage{amssymb}

% For source code output
\usepackage{xcolor}
\usepackage{minted}

\usepackage{titlesec}

\titleformat{\section}
  {\normalfont\Large\bfseries}{\thesection.}{0.3em}{}

\titleformat{\subsection}
  {\normalfont\large\bfseries}{\thesubsection.}{0.3em}{}

\titlespacing{\section}{0pt}{*2}{*2}
\titlespacing{\subsection}{0pt}{*1}{*1}
\titlespacing{\subsubsection}{0pt}{*0}{*0}
\usepackage{listings}
\lstloadlanguages{Lisp}
\lstset{extendedchars=false,
	breaklines=true,
	breakatwhitespace=true,
	keepspaces = true,
	tabsize=2
}
\begin{document}


\section*{Отчет по лабораторной работе №\,2 
по курсу \guillemotleft  Функциональное программирование\guillemotright}
\begin{flushright}
Студент группы 8О-307 МАИ \textit{Ефимов Александр}, \textnumero 7 по списку \\
\makebox[7cm]{Контакты: {\tt aleks.efimov2011@yandex.ru} \hfill} \\
\makebox[7cm]{Работа выполнена: 12.03.2021 \hfill} \\
\ \\
Преподаватель: Иванов Дмитрий Анатольевич, доц. каф. 806 \\
\makebox[7cm]{Отчет сдан: \hfill} \\
\makebox[7cm]{Итоговая оценка: \hfill} \\
\makebox[7cm]{Подпись преподавателя: \hfill} \\

\end{flushright}

\section{Тема работы}
Простейшие функции работы со списками Коммон Лисп

\section{Цель работы}
Научиться конструировать списки, находить элемент в списке, использовать схему линейной и 
древовидной рекурсии для обхода и реконструкции плоских списков и деревьев.

\section{Задание}

Вариант: №2.36

Дан список действительных чисел ($X_1 ... X_n$).
Запрограммируйте рекурсивно на языке Common Lisp функцию, которая возвращает:

\begin{itemize}
    \item  сам список, если последовательность $X_1, ..., X_n$ упорядочена по убыванию,
    т.е. $X_1 > X_2 > ... > X_n$;
    \item  список ($X_n ... X_1$) в противном случае.
\end{itemize}


\section{Оборудование студента}
Процессор Intel(R) Core(TM) i5-8250U CPU @ 1.60GHz, память: 7.6Gi, разрядность системы: 64.

\section{Программное обеспечение}
ОС Arch Linux, система CLisp.

\section{Идея, метод, алгоритм}

Проверить, если значения в списке убывают. Причем, если размер списка меньше двух, считать,
что он убывающий.

Вернуть его же, если он убывающий, или обратить иначе.

% \section{Сценарий выполнения работы}

\section{Распечатка программы и её результаты}

\subsection{Исходный код}
\inputminted[linenos, frame=lines]{lisp}{./descend-reverse.lisp}

\subsection{Результаты работы}
\inputminted[frame=lines]{lisp}{./log.lisp}

% \section{Дневник отладки}
% \begin{tabular}{|c|c|c|c|}
% \hline
% Дата & Событие & Действие по исправлению & Примечание \\
% \hline
% \end{tabular}

% \section{Замечания автора по существу работы}

\section{Выводы}
% \begin{minted}{text}
% Take me down to the Haskell city
% Where the brackets are little
% And syntax is pretty
% Oh won't you please take me home?
% \end{minted}
По спискам можно итерировать с помощью функций \textit{first} и \textit{rest}.

\end{document}

